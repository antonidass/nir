\maketableofcontents

\intro



Стремительный рост в сфере информационных технологий вызывает ряд проблем, связанных с защитой сетевых ресурсов в глобальной сети.
Так, согласно исследованиям с начала 2022 года количество сетевых атак увеличилось на $15\%$ по сравнению с 2021 годом. При этом увеличилась доля массовых атак: теперь их количество составляет 33\% от общего числа. Также в исследовании отмечается, что вырос интерес к интернет-ресурсам: доля атак на них увеличилась до 22\% от общего количества по сравнению с 13\%, наблюдаемыми в 2021 году \cite{investigate}.

На основании данного исследования можно сделать вывод, что количество сетевых атак лишь растет, а следовательно растет и потребность в защите от них.


Цель работы --- классифицировать известные методы обнаружения сетевых атак.

Чтобы достигнуть поставленной цели, требуется решить следующие задачи:

\begin{itemize}
    \item описать термины предметной области и обозначить проблему;
    \item рассмотреть возможные способы защиты от сетевых атак;
	\item классифицировать методы обнаружения сетевых атак;
	\item сформулировать критерии сравнения методов защиты от сетевых атак;
	\item сравнить описанные методы по предложенным критериям.
\end{itemize}

