\chapter{Анализ предметной области}

\textbf{Сетевая атака} \cite{second} --- это действие или последовательность связанных между собой действий, использующих уязвимости информационной системы и приводящих к нарушению политики безопасности. Под политикой безопасности подразумевается набор критериев и правил, описывающих информационные процессы в системе, выполнение которых обеспечивает необходимое условие безопасности системы.

\section{Модели сетевых атак}

\textbf{Классическая модель} атаки выстраивается по принципу  <<один к одному>>, как показано на рисунке 1.1, или <<один ко многим>>, как показано на рисунке 1.2. Для защиты от таких атак разработчики внедряют сенсоры системы защиты, передающие информацию на центральный аппарат управления.
Благодаря этому обеспечивается масштабируемость системы и простота удаленного управления. Однако такое решение не распространяется на модель с распределенными атаками \cite{third}.

\imgsvg{attack_model_1}{h!}{0.45}{12}{1}{Сетевая атака <<один к одному>>}
\clearpage
\imgsvg{attack_model_2}{h!}{0.45}{12}{1}{Сетевая атака <<один ко многим>>}


В \textbf{распределенной модели} используется принцип <<многие к одному>>, как показано на рисунке 1.3, или <<многие ко многим>>, как показано на рисунке 1.4. Данная атака осуществляется в два этапа. На первом этапе ищутся узлы сети, которые впоследствии задействуются для реализации распределенной атаки. Второй этап представляет собой посылку большого количества запросов на атакуемый сетевой узел.
Отправка запросов осу­ществляется с помощью скомпроме­тированных систем-посредников, на которых установлены специальные агенты, реализующие распределенную атаку. Агенты делятся на два типа: <<мастер>> и <<демон>>. Злоумышленник управляет небольшим числом <<мастеров>>, а те управляют <<демонами>>.
Стоит отметить, что блокирование одного или нескольких <<мастеров>> или <<демонов>> не приводит к завершению атаки \cite{third}.

\clearpage

\imgsvg{attack_model_3}{h!}{0.45}{12}{1}{Сетевая атака <<многие к одному>>}
\imgsvg{attack_model_4}{h!}{0.45}{12}{1}{Сетевая атака <<многие ко многим>>}

\section{Классификация сетевых атак}

Большинство сетевых атак нацелены на изменение определенных параметров безопасности системы. Например, с помощью некоторых атак злоумышленник может получить возможность просматривать передаваемые сообщения, но не изменять их. Другие атаки могут позволить злоумышленнику выполнить останов некоторых компонент системы, при этом не предоставляя доступ к ресурсам, хранящимся в данной системе \cite{third}.

Существует множество различных типов классификации атак. Например, деление на внешние и внутренние, пассивные и активные, умышленные и неумышленные.
Однако, все сетевые атаки можно разделить на два класса: пассивные и активные \cite{third}.

\textbf{Пассивная атака} \cite{fourth} --- это атака, при которой у злоумышленника нет доступа к модификации передаваемых сообщений и возможности добавления собственных сообщений в информационный канал между отправителем и получателем.
Основная цель пассивной атаки --- прослушивание передаваемых сообщений и анализ сетевого трафика.

\imgsvg{passive}{h!}{0.65}{12}{1}{Пассивная атака}

Одной из разновидностей пассивных атак являются \textbf{атаки сканирования} \cite{fifth}. 
Атаки сканирования не нацелены на проникновение в систему. Они помогают злоумышленнику определять:

\begin{itemize}
    \item топологию сети;
    \item активные хосты в сети;
	\item выполняемое на хостах ПО сервера;
	\item номера версий обнаруженного ПО.
\end{itemize}

На рисунке 1.6 представлены инструментальные средства для осуществления атаки сканирования.

\imgsvg{types}{h!}{0.8}{10}{1}{Инструментальные средства для осуществления атаки сканирования}



\textbf{Активная атака} \cite{six} --- это атака, при которой у злоумышленника имеется возможность модифицировать передаваемые сообщения и добавлять собственные. Существуют следующие типы активных атак:


\begin{enumerate}

    \item \textbf{Отказ в обслуживании --- DoS-атака (Denial of Service)} \cite{fifth}. Отказ в обслуживании нарушает функционирование сетевых сервисов. Суть данной атаки заключается в следующем: на сетевой сервис поступает значительное количество запросов, в результате чего сетевой сервис перестает обрабатывать запросы реальных клиентов, а злоумышленник может перехватывать все сообщения направленные определенному адресату. DoS-атака базируется на классической модели атак.  \clearpage  \imgsvg{dos}{h!}{0.4}{10}{1}{DoS-атака}
    \item \textbf{Распределенный отказ в обслуживании --- DDoS (Distributed Denial of Service)} \cite{fifth}. Основным отличием DDoS-атаки от DoS-атаки является использование распределенной модели атак.
    \item \textbf{Модификация потока данных --- MITM-атака (Men In The Middle)} \cite{six}. MITM-атака --- это атака, с помощью которой злоумышленник перехватывает связь между клиентом и сервисом. Позиционируя себя между законным клиентом и сервисом, злоумышленник может отключить шифрование и перехватить сообщения, отправляемые клиенту или сервису. Это позволяет злоумышленнику получать конфиденциальную информацию, такую как учетные данные и другую личную информацию.  \imgsvg{mitm}{h!}{0.4}{10}{1}{MITM-атака}
    \item \textbf{Создание ложного потока (фальсификация)} \cite{seventh}. Фальсификация означает попытку одного субъекта выдать себя за другого. С помощью этой атаки злоумышленник может получить привилегии, которые не предусмотрены данной системой. Привилегия позволяет злоумышленнику в дальнейшем нарушить конфиденциальность, доступность или целостность сервиса.
    \item \textbf{Атака повторного воспроизведения --- Replay-атака} \cite{seventh}. Атака повторного воспроизведения --- атака на систему аутентификации путём записи и последующего воспроизведения ранее посланных корректных сообщений или их частей. Для совершения данного вида атаки злоумышленник пользуется несовершенством системы аутентификации потока данных. Злоумышленник перехватывает несколько пакетов или команд приложения, изменяет их и воспроизводит с целью выполнения несанкционированных действий. 
\end{enumerate}


\imgsvg{replay}{h!}{0.4}{10}{1}{Replay-атака}


\chapter{Методы обнаружения сетевых атак}

% Для классификации способов обнаружения сетевых атак буду рассмотрены следующие методы: Цепи Маркова, Метод $\chi^2$, Метод среднеквадратических отклонений, Анализ распределений интенсивности передачи пакетов, Анализ временных рядов, Пороговый анализ.

Общий алгоритм выявления сетевых атак описывается следующим образом: 

\begin{enumerate}
    \item Собирается весь сетевой трафик, представленный как набор сетевых пакетов.
    \item Вычисляются признаковые атрибуты сетевого трафика и строится профиль активности пользователя.
    \item Созданный набор признаковых атрибутов сравнивается с набором характеристик нормального поведения пользователя.
    \item Если в результате получилось весомое расхождение сравниваемых атрибутов, то фиксируется сетевая атака. В противном случае происходит изменение параметров нормального поведения.
\end{enumerate}


\section{Обзор методов обнаружения сетевых атак}

\subsection*{Пороговый анализ}

В методе порогового анализа \cite{eight} сначала выбирается набор сетевых параметров, а именно:

\begin{itemize}
    \item IP-адреса источника и приемника --- \textbf{S/D};
    \item тип и порт пакета --- \textbf{Tp};
    \item длина пакета --- \textbf{L};
    \item время фиксации пакета --- \textbf{Tm}.
\end{itemize}

Следовательное любое зафиксированное событие в сети можно описать вектором-объектом события $Tr = <S/D,  Tp, L, Tm>$.
Далее из события $Tr$ извлекается объект $X = Tr<L>$, который соответствует длине зафиксированного пакета. Пусть $X_i$ --- событие из множества событий $X$, в некоторый момент времени. Тогда $Y_i$ --- аналогичный набор событий из множества, составляющего шаблон штатного функционирования сети. Затем выбирается коэффициент <<чувствительности>> $k = 0.8$. Тогда нижний порог определяется как $X_i > k Y_i$, а верхний как $X_i < \frac{Y_i}{k}$. После этого определяются краевые значения допустимых интервалов. Для этого берется выборочное среднее $\bar{X}$.

\begin{equation}
    \bar{X} = \frac{1}{n} \sum_{i = 0}^{n} X_i
\end{equation}

Допустимый диапазон определяется следующим неравенством: 

\begin{equation}
    \frac{\bar{X}}{2} < X_i <  \frac{3}{2} \bar{X}
\end{equation}

Нахождение вне рамок этого диапазона свидетельствует аномальному поведению.

Недостатком данного метода является необходимость точного задания коэффициента <<чувствительности>> $k$ и отсутствие адаптивных механизмов для автоматического выбора порога \cite{fifthteen}.



\subsection*{Анализ энтропии}

Как известно энтропия множества $X$ определяется следующим образом: 

\begin{equation}
    H(X) = - \sum_{i = 1}^{n} p_i \log_2 p_i,
\end{equation}

где $p_i$ --- вероятность $i$-го состояния системы, $n$ --- количество всех возможных состояний системы \cite{fourteen}.

Метод анализа энтропии \cite{nine} базируется на построении модели, которая максимизировала бы значение энтропии. Так как с увеличением количества уникальных записей происходит их равномерное распределение относительно классов множества $X$, что приводит к увеличению энтропии. 

Для анализа энтропии выбирается следующий набор сетевых параметров 

\begin{itemize}
    \item IP-адрес источника;
    \item IP-адрес приемника;
    \item сетевой порт источника;
    \item сетевой порт приемника.
\end{itemize}

\textbf{Алгоритм}

\begin{enumerate}
    \item Выбираются атрибуты для построения энтропийных временных рядов.
    \item Строится множество временных рядов $T$.
    \item Для каждого $T_i$ определяется ошибка прогноза на момент времени $t$: 
    
    \begin{equation}
        \delta_i = | Pred(T_i(t)) - T_i(t) |.
    \end{equation}
    
    \item Нормализуются  ошибки предсказания относительно дисперсии соответствующих временных рядов путем умножения на весовой коэффициент. Весовой коэффициент вычисляется по следующей формуле:
    
    \begin{equation}
        \omega_i = \frac{1}{\sigma_i} max(\sigma_1,...,\sigma_n).
    \end{equation}

    \item Вводится совокупная характеристика AS (anomaly score):
    \begin{equation}
        AS = \sum_{i = 1}^{n} \delta_i \omega_i
    \end{equation}

    \item Если $AS > AS_{thr}$, то фиксируется некая аномалия сетевого трафика. Пороговая величина $AS_{thr}$ определяется эмпирически в зависимости от количества базовых временных рядов $n$.
\end{enumerate}


\subsection*{Байесовский метод}

Байесовская сеть \cite{ten} представляет собой модель, которая кодирует вероятностные отношения между некоторыми событиями и предоставляет механизм для вычисления условных вероятностей их наступления. В данном методе используются оценочные функции для определения вероятностей новых сетевых атак. Вследствие свойств предложенного метода системе не нужны предварительные знания о шаблонах атак.

\textbf{Алгоритм}

На первом этапе собирается информация о сетевых параметрах, а именно: 

\begin{itemize}
    \item IP-адрес источника;
    \item IP-адрес приемника;
    \item сетевой порт источника;
    \item сетевой порт приемника;
    \item состояние соединения;
    \item временная метка.
\end{itemize}

Далее применяются правила ассоциации $X -> Y$ к записям соединений, где $X$ и $Y$ предусловие и постусловие правил, описанных внутри ядра системы соответственно. 
Затем строятся профили нормального поведения клиентов системы и генерируются правила ассоциации, используемые впоследствии для обучения.

Основным преимуществом данного метода является работа в режиме реального времени.


\subsection*{SVM-метод}

Метод опорных векторов --- SVM-метод (Support Vector Machine) \cite{eleven} рассматривается как один из ключевых методов обнаружения вторжений. 
SVM является производным от линейно разделяемой гиперплоскости оптимальной классификации, и его основная идея может быть объяснена двумерным случаем представленным на рисунке 2.1.
Существует обучающий набор $D = {(X_1, y_1), (X_2, y_2),...,(X_n, y_n)}$, где $X_i$ --- характеристический вектор обучающей выборке и $y_i$ --- соответствующая метка класса. $y_i$ принимает значения $+1$ или $-1$, указывая, принадлежит вектор к этому классу или нет. Говорят, что он линейно разделим, если существует линейная функция, которая может полностью разделить на два класса. В противном случае он нелинейно разделим.

Рисунок 2.1 представляет собой линейно разделяемый случай, поскольку можно провести прямую линию, чтобы отделить вектор класса $+1$ от вектора класса $-1$.
Существует бесчисленное множество таких линий, и так называемая оптимальная линия классификации требующая, чтобы две выборки были правильно разделены и чтобы интервал разделения был наибольшим. SVM завершает классификацию выборки путем поиска той, которая имеет наибольший интервал классификации.

\imgsvg{svm}{h!}{0.8}{10}{1}{SVM-метод}

Оптимальная линия классификации может быть выражена уравнением $\omega x + b = 0 \space (\omega \in R^n, b \in R)$, $\omega$ представляет собой вектор веса, а $b$ --- скаляр, называемый смещением. Точки над разделяющей гиперплоскостью удовлетворяются следующим образом:

\begin{equation}
    \omega x + b > 0.
\end{equation}

Аналогично, точки ниже разделяющей гиперплоскости удовлетворяются следующим образом:

\begin{equation}
    \omega x + b < 0.
\end{equation}

Мы можем отрегулировать вес, чтобы крайняя сторона гиперплоскости могла быть выражена как:

\begin{equation}
    \begin{split}
    H1 : \omega x + b \ge 1, для y_i = 1; \\
    H2 : \omega x + b \le 1, для y_i = -1.
    \end{split}
\end{equation}

Это означает, что векторы, падающие на или выше $H1$, принадлежат классу $+1$, а векторы, падающие на или ниже $H2$, принадлежат классу $-1$.

Обнаружение сетевой атаки эквивалентно задаче с двумя классификациями. Сначала собираются данные сетевого подключения для обучения, затем находится оптимальная классификационная гиперплоскость между данными нормального поведения и данными сетевой атаки \cite{thirteen}.



\section{Сравнение и оценка методов}

Сравнение методов обнаружения сетевых атак произведено по следующим критериям:

\begin{itemize}
    \item К1 --- адаптивность;
    \item К2 --- устойчивость;
    \item К3 --- уровень наблюдения;
\end{itemize}

Результаты сравнений приведены в таблице \ref{tbl:compare}.

% Настройка выравнивание колонки по центру
\newcolumntype{P}[1]{>{\centering\arraybackslash}p{#1}}

\begin{center}
    \captionsetup{justification=raggedleft,singlelinecheck=off}
    \begin{longtable}[c]{|P{3cm}|P{3cm}|P{3cm}|P{3cm}|}
    \caption{Сравнение методов\label{tbl:compare}}
    \\ \hline
        & 
        \textbf{К1} &
        \textbf{К2} &
        \textbf{К3} 
    \\ \hline
        \textbf{Анализ энтропии} &
        + &
        - &
        HIDS, NIDS, AIDS, Hybrid 
    \\ \hline
        \textbf{Пороговый анализ} &
        + &
        - &
        HIDS, NIDS, AIDS, Hybrid 
    \\ \hline
        \textbf{Байесовский метод} &
        - &
        + &
        NIDS, HIDS
    \\ \hline
        \textbf{SVM-метод} &
        + &
        - &
        NIDS, HIDS 
    \\ \hline
\end{longtable}
\end{center}

В таблице используется обозначения уровня наблюдения появления аномалии в сети:

\begin{itemize}
    \item HIDS \cite{twelve} --- наблюдение на уровне операционной системы отдельного узла сети;
    \item NIDS \cite{twelve} --- наблюдение на уровне сетевого взаимодействия объектов на узлах;
    \item AIDS \cite{twelve} —-- наблюдение на уровне отдельных приложений узла сети;
    \item Hybrid \cite{twelve} —-- комбинация наблюдателей разных уровней.
\end{itemize}

\section{Вывод}

Таким образом, по результатам сравнительного анализа было выявлено, что методы порогового анализа и анализа энтропии обладают наиболее высокой адаптивностью к новым данным и способны проводить наблюдение на всех выделенных уровнях сети.  


% @online{gs1-128,
%   language = "russian",
%   title    = "СТАНДАРТ ГС1 РУС [Электронный ресурс]",
%   note     = "URL: \url{https://www.gs1ru.org/wp-content/uploads/2017/02/СТО-30_V_1_открыт.pdf} (дата обращения: 20.11.2022)"
% }
